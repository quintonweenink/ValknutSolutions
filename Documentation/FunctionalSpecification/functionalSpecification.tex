\documentclass{article}

\usepackage[margin=2.5cm,left=2cm,includefoot]{geometry}
\usepackage{graphicx}
\usepackage{float}
\usepackage[space]{grffile}
\usepackage{hyperref}
\usepackage[export]{adjustbox}
\usepackage{multicol}
\usepackage{caption}
\usepackage{hyperref}
\usepackage{listings}
\usepackage{vhistory}
\newcommand{\nexists}{\not\exists}
\newcommand\tab[1][0.5cm]{\hspace*{#1}}

% Header and footer
\usepackage{fancyhdr}
\pagestyle{fancy}

\rhead{COS301}
\lhead{Functional Specification}
\fancyfoot[R]{Page \thepage}

\renewcommand{\headrulewidth}{2pt}
\renewcommand{\footrulewidth}{1pt}

\begin{document}

	\begin{titlepage}
		\begin{center}
			\includegraphics[width=10cm]{images/UP.jpg}  \\
			[0.5cm]
			\huge{
			Functional Specification\\
			}

			\line(1,0){300}\\
			[0.2cm]
			\LARGE{Project: Insurance profiling from social media\\
			Client: RetroRabbit} \\
			\line(1,0){300}\\
			\LARGE{Team: Valknut Solutions}\\
			[1.0cm]
			\large
			{
			\begin{itemize}
				\item 13054903 - Charl Jansen van Vuuren
				\item 13044924 - Kevin Heritage
				\item 13176545 - Quinton Weenink\\
			\end{itemize}
			}
			\textsc{\large}\\
		[3.0cm]
		\textsc{\large  Department of Computer Science}\\
		[0.5cm]
		\textsc{\large \today}\\
		\end{center}

		%\begin{figure}[H]}
		%\centering
		%\includegraphics[{imagename}
		%\end{figure}\

	\end{titlepage}
	\cleardoublepage
	\tableofcontents
	\cleardoublepage
	% Start of the revision history table
	\begin{versionhistory}
  		\vhEntry{1.0}{27.05.2016}{CJvV,KH,QW}{Original Architectural document without Functional requirements}
  		\vhEntry{1.1}{27.07.2016}{CJvV,KH,QW}{Added Functional requirements and separated Architectural requirements}
		\vhEntry{1.2}{27.07.2016}{CJvV,KH,QW}{Added better scoping requirements, updated use cases, updated each subsystem}
	\end{versionhistory}
	
	\pagebreak

\section{Functional requirements and application design}
	\subsection{Overall System Scope}
	The system consists of five modularized subsystems namely:
	\begin{itemize}
		\item Administration Section \ref{subsec:Admin}
		\item Notifications Section \ref{subsec:Notifcations}
		\item Analysis Section \ref{subsec:Analysis}
		\item Persistence with regards to data Section \ref{subsec:Persistence}
		\item Social media Section \ref{subsec:SocialMedia}
	\end{itemize}
		\begin{figure}[H]
		\includegraphics[width=\textwidth]{images/uc__insuranceProfiling__scope.jpg}  \\
		\caption{Scope : Insurance Profiling}
		\end{figure}
		
		
		\pagebreak
	\subsection{Social Media subsystem}\label{subsec:SocialMedia}
	The social media subsystem forms a major part of the system's functionality. The included modules handle the receiving of data from the Facebook advertisement lead form, and the other integration sources.
		\subsubsection{Use cases}
		\begin{figure}[H]
		\includegraphics[width=\textwidth]{images/uc__socialMedia__scope.jpg}  \\
		\caption{Use Case Diagram : Social Media}
		\end{figure}

		\begin{flushleft}
			\textbf{Critical}
				\begin{itemize}
	  				\item validateCustomer
	  				\item createCustomer
	  				\item getCustomer
				\end{itemize}
			\textbf{Important}
				\begin{itemize}
	  				\item validateCustomer
				\end{itemize}
			\textbf{Nice-To-Have}
				\begin{itemize}
	  				\item getAnalyst
	  				\item analyseUser
				\end{itemize}
		\end{flushleft}

		\subsubsection{Services Contracts}

		\begin{figure}[H]
		\includegraphics[width=\textwidth]{images/class__getLeadData__serviceContract.jpg}  \\
		\caption{Service Contract : getLeadData}
		\end{figure}

		\subsubsection{Required Functionality}

		\begin{figure}[H]
		\includegraphics[width=\textwidth]{images/obj__Required_Functionality__addLeadData.jpg}  \\
		\caption{Required Functionality : getLeadData}
		\end{figure}

		\subsubsection{Process specifications}

		\begin{figure}[H]
		%\includegraphics[width=\textwidth]{images/Incomplete.png}  \\
		\caption{Process specification : Publications}
		\end{figure}
		
	\pagebreak
	\subsection{Analysis subsystem}\label{subsec:Analysis}
	The analysis subsystem handles another core functionality of the system, analysis of the retrieved data.\\ This includes the ability to generate reports in the form of different graphs, filtered by different fields.\\ Analysis and reporting forms a major feature from a marketing and risk analysis standpoint.
		\subsubsection{Use cases}

		\begin{figure}[H]
		\includegraphics[width=\textwidth]{images/uc__analysis__scope.jpg}  \\
		\caption{Use Case Diagram : Analysis}
		\end{figure}

		\begin{flushleft}
			\textbf{Critical}
				\begin{itemize}
					\item getUsers
					\item generateReport
					\item drawGraph
				\end{itemize}
			\textbf{Important}
				\begin{itemize}
				\item setGraphType
				\end{itemize}

			\textbf{Nice-To-Have}
				\begin{itemize}
					\item filterGraph
				\end{itemize}
		\end{flushleft}

		\subsubsection{Services Contracts}

		\begin{figure}[H]
		\includegraphics[width=\textwidth]{images/class__drawGraph__serviceContract.jpg}  \\
		\caption{Service Contract : drawGraph}
		\end{figure}

		\subsubsection{Required Functionality}

		\begin{figure}[H]
		%\includegraphics[width=\textwidth]{images/obj__Required_Functionality__addLeadData.jpg}  \\
		\caption{Required Functionality : drawGraph}
		\end{figure}

		\subsubsection{Process specifications}

		\begin{figure}[H]
		%\includegraphics[width=\textwidth]{images/Incomplete.png}  \\
		\caption{Process specification : drawGraph}
		\end{figure}
	\pagebreak
	\subsection{Administration subsystem}\label{subsec:Admin}
	The administration subsystem handles the authentication and management of the analysts and administrators of the system.
		\subsubsection{Use cases}
		\begin{figure}[H]
		\includegraphics[width=\textwidth]{images/uc__administration__scope.jpg}  \\
		\caption{Use Case Diagram : Administration}
		\end{figure}

		\begin{flushleft}
			\textbf{Critical}
				\begin{itemize}
					\item createAdministrator
					\item createAnalyst
					\item login
					\item getUser
				\end{itemize}
			\textbf{Important}
				\begin{itemize}
					\item removeAnalyst
					\item removeAdministrator
				\end{itemize}

			\textbf{Nice-To-Have}
				\begin{itemize}
					\item logout
					\item analiseUser
				\end{itemize}
		\end{flushleft}

		\subsubsection{Services Contracts}

		\begin{figure}[H]
		\includegraphics[width=\textwidth]{images/class__createAdministrator__serviceContract.jpg}  \\
		\caption{Service Contract : createAdministrator}
		\end{figure}

		\begin{figure}[H]
		\includegraphics[width=\textwidth]{images/class__login__serviceContract.jpg}  \\
		\caption{Service Contract : login}
		\end{figure}

		\subsubsection{Required Functionality}

		\begin{figure}[H]
		%\includegraphics[width=\textwidth]{images/obj__Required_Functionality__addLeadData.jpg}  \\
		\caption{Required Functionality : createAdministrator}
		\end{figure}

		\subsubsection{Process specifications}

		\begin{figure}[H]
		%\includegraphics[width=\textwidth]{images/Incomplete.png}  \\
		\caption{Process specification : createAdministrator}
		\end{figure}

	\pagebreak
	\subsection{Persistence subsystem}\label{subsec:Persistence}
	The persistence subsystem handles the throughout persisting of information in the system. Persistence in this case related to the information passing between database and system.
		\subsubsection{Use cases}

		\begin{figure}[H]
		\includegraphics[width=\textwidth]{images/uc__persistence__scope.jpg}  \\
		\caption{Use Case Diagram : Persistence}
		\end{figure}

		\begin{flushleft}
			\textbf{Critical}
				\begin{itemize}
					\item persistAdministrator
					\item persistAnalyst
					\item persistUser
				\end{itemize}
		\end{flushleft}

		\subsubsection{Services Contracts}

		\begin{figure}[H]
		%\includegraphics[width=\textwidth]{images/class__createAdministrator__serviceContract.jpg}  \\
		\caption{Service Contract : persistUser}
		\end{figure}

		\subsubsection{Required Functionality}

		\begin{figure}[H]
		%\includegraphics[width=\textwidth]{images/obj__Required_Functionality__addLeadData.jpg}  \\
		\caption{Required Functionality : persistUser}
		\end{figure}

		\subsubsection{Process specifications}

		\begin{figure}[H]
		%\includegraphics[width=\textwidth]{images/Incomplete.png}  \\
		\caption{Process specification : persistUser}
		\end{figure}

	\pagebreak
	\subsection{Notifications subsystem}\label{subsec:Notifcations}
	The notification subsystem handles the ability to send responses to customers and analysts. The use case might seem simplified, but this design allows for different notification channels. The notify module can be specialized in the format of Email, SMS and JSON responses or any other accessible notification module.
		\subsubsection{Use cases}

		\begin{figure}[H]
		\includegraphics[width=\textwidth]{images/uc__notifications__scope.jpg}  \\
		\caption{Use Case Diagram : Notifications}
		\end{figure}

		\begin{flushleft}
			\textbf{Critical}
				\begin{itemize}
					\item Notify
				\end{itemize}
			\textbf{Important}
				\begin{itemize}
					\item Email
				\end{itemize}
			\textbf{Nice-to-have}
				\begin{itemize}
					\item SMS
					\item JSON Objects
				\end{itemize}
		\end{flushleft}

		\subsubsection{Services Contracts}

		\begin{figure}[H]
		\includegraphics[width=\textwidth]{images/class__notify__serviceContract.jpg}  \\
		\caption{Service Contract : notify}
		\end{figure}

		\subsubsection{Required Functionality}

		\begin{figure}[H]
		%\includegraphics[width=\textwidth]{images/obj__Required_Functionality__addLeadData.jpg}  \\
		\caption{Required Functionality : notify}
		\end{figure}

		\subsubsection{Process specifications}

		\begin{figure}[H]
		%\includegraphics[width=\textwidth]{images/Incomplete.png}  \\
		\caption{Process specification : notify}
		\end{figure}


\subsection{Domain model}

\begin{figure}[H]
\includegraphics[width=\textwidth]{images/class__analysis__domainModel.jpg}  \\
\caption{Domain Model : Analysis}
\end{figure}

\begin{figure}[H]
\includegraphics[width=\textwidth]{images/class__leadAdd__domainModel.jpg}  \\
\caption{Domain Model : Social Media}
\end{figure}

\subsection{Open Issues}






\end{document}
