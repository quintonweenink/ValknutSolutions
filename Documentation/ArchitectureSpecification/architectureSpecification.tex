\documentclass{article}

\usepackage[margin=2.5cm,left=2cm,includefoot]{geometry}
\usepackage{graphicx}
\usepackage{float}
\usepackage[space]{grffile}
\usepackage{hyperref}
\usepackage[export]{adjustbox}
\usepackage{multicol}
\usepackage{caption}
\usepackage{hyperref}
\usepackage{listings}
\usepackage{vhistory}
\usepackage{titlesec}

\setcounter{secnumdepth}{4}

\titleformat{\paragraph}
{\normalfont\normalsize\bfseries}{\theparagraph}{1em}{}
\titlespacing*{\paragraph}
{0pt}{3.25ex plus 1ex minus .2ex}{1.5ex plus .2ex}

% Header and footer
\usepackage{fancyhdr}
\pagestyle{fancy}

\rhead{COS301}
\lhead{Architecture Requirements}
\fancyfoot[R]{Page \thepage}

\renewcommand{\headrulewidth}{2pt}
\renewcommand{\footrulewidth}{1pt}

\begin{document}

	\begin{titlepage}
		\begin{center}
			\includegraphics[width=10cm]{images/UP.jpg}  \\
			[0.5cm]
			\huge{
			Architectural Requirements, Initial Architecture Design and Non-Functional Requirements\\
			}
			
			\line(1,0){300}\\
			[0.2cm]
			\LARGE{Project: Insurance profiling from social media\\
			Client: RetroRabbit} \\
			\line(1,0){300}\\
			\LARGE{Team: Valknut Solutions}\\
			[1.0cm]
			\large
			{
			\begin{itemize}
				\item 13054903 - Charl Jansen van Vuuren 
				\item 10297902 - Bernhard Schuld      
				\item 13044924 - Kevin Heritage
				\item 13176545 - Quinton Weenink\\
			\end{itemize}
			}
			\textsc{\large}\\
		[3.0cm]
		\textsc{\large  Department of Computer Science}\\
		[0.5cm]
		\textsc{\large \today}\\
		\end{center}

		%\begin{figure}[H]}
		%\centering
		%\includegraphics[{imagename}
		%\end{figure}\
			
	\end{titlepage}
	\cleardoublepage
	\tableofcontents
	\cleardoublepage
	
		% Start of the revision history table
	\begin{versionhistory}
  		\vhEntry{1.0}{27.05.2016}{CJvV,KH,QW}{Created}
  		\vhEntry{1.1}{27.07.2016}{CJvV,KH,QW}{Introduction, Scope, Vision changes}
	\end{versionhistory}	


%Quinton addition (Maybe just have a look at this and see if you can use it) This should be slightly more revised

\section{Introduction}
 Insurance profiling and risk analysis often require large amounts of data to generate in-depth profiles, our project aims to eliminate the need for vast amounts of data gathering by utilising a user's social media information, based on advertisement form.

\section{Scope of system} 
The project as it is currently specified is an application that will allow insurance companies the ability to get data from lead data provided via lead advertisements on Facebook.
A user would fill in a Facebook advertisement to purchase short term insurance based on Facebook's advertisement analytics. \\
This data will be used to charge the users for the insurance that they have signed up for while also providing the application with some basic data that can later be used to analyse the users and the data they provided. \\
An analyst would be able to log into our system and generate risk statistics for a certain advertisement based on certain criteria, this information gives the insurance company the ability to charge and adapt their sale and costs based on popular advertisements.\\
This information will be available based on API calls to the database or through a Web front-end for analysts.

\section{Vision}
As the project is currently specified the ambition is to allow the analyst the ability to get the best indication of the risk associated with each user.\\
The primary focus of this project is to generate risk statistics based on social media advertisements.


\section{Architectural Requirements}
	\subsection{Access channel requirements}
	\begin{enumerate}
		\item A user would log in to the system via a Facebook Login button hosted on our website. The Facebook login is achieved via a REST API call to Facebook's servers.
		\item A login dialogue will ask the user to log into their Facebook account and will redirect to a permission request afterwards. The permission request specifies what data the system will gather from the user's Facebook account.
		\item The necessary data will be gathered and analysed by the system.
		\item A report will be generated which will be displayed to the user via a web page.
		\item Further options will be accessible via this web page. 
		\item Integration with Facebook advertisements will also be considered, as this forms part of a different method of access to the system, not through our website, but through the user's Facebook dashboard.
	\end{enumerate}
	Further access channel specification include:
	\begin{itemize}
		\item The website will be accessible from, and optimised for all web-browsers including mobile phone browsers.
		\item The possibility of developing a mobile application will be considered as per the client's request.
		\item The website will be accessible via the majority of operating systems if such an operating system has access to a supported web-browser.
		\item Facebook advertisement integration will be accessible via the normal Facebook access channels.
	\end{itemize}
	 
	
	\subsection{Quality Requirements}
		\subsubsection{Performance}
		\begin{itemize}
			\item A user should be able to request a quote in less than 5 minutes. This is easily achieved by means of the Facebook login and will only be user-network dependent.
			\item Once the REST API call is made, it saves the user's data in our database, increasing the efficiency of future requests and processing of the data.
		\end{itemize}
		 
		\subsubsection{Security}\label{subsubsec:security}
		\begin{itemize}
			\item Security is our most important architectural requirement. A user's personal information is used to generate these risk profiles and as a result the user will trust that this information is not shared with other parties, and properly secured.
			\item Only authorised persons will have access to the generated profile and access to the database will need to be restricted to the highest authority.
		\end{itemize}
		\subsubsection{Scalability}
		\begin{itemize}
			\item Since this project is a web-based solution, the possibility of multiple concurrent users should be considered.
			\item The server should account for a vast amount of concurrent users.
			\item As per integratability, the ability to change the risk analysis algorithm as needed should be considered for a future scalable solution.
		\end{itemize}

		\subsubsection{Integratability}\label{subsubsec:integratability}
		\begin{itemize}
			\item The project will integrate with Facebook and utilise it as the primary data provider.
			\item A Facebook approved login button on our website will allow access to the information requested. 
			\item The ability to integrate with other social media platforms should be considered and modularised accordingly, to ensure seamless future integration. 
			\item The ability to change the risk analysis algorithm as needed should be considered.
			\item Integration includes the connection from the website to the profiling engine and back to the website as a report.
			\item Further integration might include the use of Facebook's advertisements. Instead of logging into our website the user will be able to generate a report directly from their Facebook dashboard.
		\end{itemize}		

		\subsubsection{Reliability}
			Since our solution is mainly web-based, the platform as a service (PaaS) offered by Heroku will ensure the website is always up to date and reliable.

		\subsubsection{Maintainability}
			The system will make use of a database with massive amounts of data. To ensure optimal performance, this data will need to be maintained and normalised on a regular basis.

		\subsubsection{Auditability}
		\begin{itemize}
			\item All actions performed in the system should be traceable to the user that performed them.
			\item The user's IP  will be logged as to have a form of accountability in the persistence of the data.
		\end{itemize}

		\subsubsection{Cost}
			The majority of our platform is open-source, except for:
		\begin{itemize}
			\item Heroku hosting if the commercial version is used
			\item Travis CI if the commercial version is used
		\end{itemize}

		\subsubsection{Usability}
		\begin{itemize}
			\item The platform is being developed with efficiency in mind, as a result the input and response of the website should be visually pleasing and simple to use.
			\item The Facebook login API aims to improve efficiency for the user as this eliminates the need to manually fill in various fields on a form.
			\item Further integration with the Facebook ad system will increase usability even more, as the user will be able to generate a report from their Facebook dashboard, without the need to log in to our website.
		\end{itemize}
		\subsubsection{Flexibility}
		\begin{itemize}
			\item The system must be usable on any internet browser (including mobile phone browsers). 
			\item The system must include the ability to change the risk analysis algorithm. Insurance profilers can use this to their advantage to generate more thorough risk profiles. 
		\end{itemize}
		
		 

	\subsection{Integration requirements}
		\subsubsection{API specifications}
			The use of API will be the API provided by Facebook. Depending on which technology our client requires us to use we will use the Facebook API for that technology.

			Facebook has APIs for almost every technology/language. If they do not have an API for that technology/language they refer you to third party plugins for the technology or language that you want to use.

			There is always the method of using the standard HTTP web request to gain access to the Facebook servers. The only thing left in this method of interacting with the API will be to parse the JSON string that will be returned by the Facebook server. It is also possible to get the access token using this method (a string of characters giving you access to a persons facebook account with the permissions specified).

			Regarding the access token, a User Access Token should be used. This User Access Token should also have the longest lifetime possible to be able to update user data for as long as possible.

			Seeing as we are using technologies that do not support the SDKs (Android, IOS, Javascript) we will be building a login flow with the use of \href{https://developers.facebook.com/docs/facebook-login/manually-build-a-login-flow}{redirects}.\\

			Building a login flow
			\begin{itemize}
				\item \href{https://developers.facebook.com/docs/facebook-login/manually-build-a-login-flow#checklogin}{Check login status}
					\begin{itemize}
						\item We have to create our own way of checking if a user is logged in (ie. an indicator)
						\item When there is no indicator, a user is presumed to be logged out.
						\item If a user is logged out, the system should ask them to log back in.
					\end{itemize}
				\item \href{https://developers.facebook.com/docs/facebook-login/manually-build-a-login-flow#login}{Logging people in}
					\begin{itemize}
						\item Invoking the login dialog
							\begin{itemize}
								\item
									\begin{verbatim}
									https://www.facebook.com/dialog/oauth?
									client_id={app-id}
									&redirect_uri={redirect-uri}
									&response_type=code&state={app-generated-string}
									\end{verbatim}
								\item An extra parameter to look into will be the `state' parameter, to guard against Cross-site Request Forgery.
							\end{itemize}
						\item Handling login dialog response
							\begin{itemize}
								\item The access token will be a parameter that is appended to the redirect URL
								\begin{verbatim}
									{redirect-URI}#access_token=ACCESS_TOKEN...
								\end{verbatim}
								\item The access token parameter is then read from the URL and can be used to fetch the data
							\end{itemize}
						\item Canceled login
							\begin{itemize}
								\item When the user cancels the login process the redirect URI will have these parameteres
								\begin{verbatim}
									REDIRECT_URI?
  									error_reason=user_denied
  									&error=access_denied
  									&error_description=The+user+denied+your+request.
								\end{verbatim}
							\end{itemize}
					\end{itemize}

				\item \href{https://developers.facebook.com/docs/facebook-login/manually-build-a-login-flow#confirm}{Confirming identity}
					\begin{itemize}
						\item Because we are using redirecting of URLs, the data could be tampered with on the client network (insecure WiFi network) \& made-up fragments or parameters can be recieved.
						\item We need to confirm the response came from the same user we were talking to.
						\item The 'code' that we recieve will have to be exchanged for an access token to the user's profile.
							\begin{verbatim}
								https://graph.facebook.com/v2.3/oauth/access_token?
    								client_id={app-id}
   									&redirect_uri={redirect-uri}
   									&client_secret={app-secret}
   									&code={code-parameter}
							\end{verbatim}
							
						Response recieved if successful:
							\begin{verbatim}
								{
								  "access_token": {access-token}, 
								  "token_type": 	{type},
								  "expires_in":	{seconds-til-expiration}
								}
							\end{verbatim}
					\end{itemize}
				\item \href{https://developers.facebook.com/docs/facebook-login/manually-build-a-login-flow#token}{Storing access tokens \& login status}
					\begin{itemize}
						\item At this point we can make API calls on behalf of the user to retrieve their data from Facebook.
						\item During this step we should save the users access token to be able to refresh their data at intervals.
					\end{itemize}
				\item \href{https://developers.facebook.com/docs/facebook-login/manually-build-a-login-flow#logout}{Logging people out}
					\begin{itemize}
						\item When the user is logged out from Facebook for any reason we should delete the access token from the database as it will become useless to us.
					\end{itemize}
				\item On a side note, we should never reveal our App-secret to the user. It should only be used on the server side.
			\end{itemize}

			\href{https://developers.facebook.com/docs/facebook-login/access-tokens}{Access Tokens}
			\begin{itemize}
				\item The access tokens generated via web logins are short-lived, but one can convert them to long-lived tokens.
				\item Apps with Standard access to Facebook's Marketing API when using long-lived tokens will recieve long-lived tokens that don't have an expiry time.
				\item For security to validate some data we may need to gain App access tokens. These tokens are only for the developers of the application not for general public.
				\item To generate an app access token:
					\begin{verbatim}
						/oauth/access_token?
					     client_id={app-id}
					    &amp;client_secret={app-secret}
					    &amp;grant_type=client_credentials
					\end{verbatim}
				\item Once again the app secret should be kept safe and hidden from users
			\end{itemize}

		\subsubsection{Protocols}
		The system will include the use of these protocols:
		\begin{itemize}
			\item HTTP/HTTPS - (Secure) Hypertext Transfer Protocol
			\item TCP/IP - Transmission Control Protocol/Internet Protocol
			\item FTP(Possibly) - File Transfer Protocol
			\item SSL - Secure Socket Layer 
		\end{itemize}
		The request for comment pages of these protocols can be accessed via:
		\begin{itemize}
			\item\href{https://tools.ietf.org/html/rfc2616}{HTTP}
			\item\href{https://tools.ietf.org/html/rfc2660}{HTTPS}
			\item\href{https://www.ietf.org/rfc/rfc793.txt}{TCP}
			\item\href{http://www.ietf.org/rfc/rfc0791.txt}{IP}
			\item\href{https://www.ietf.org/rfc/rfc959.txt}{FTP}
			\item\href{https://tools.ietf.org/html/rfc6101}{SSL}
		\end{itemize}
		\subsubsection{Integration Quality Requirements}
		As mentioned in \ref{subsubsec:integratability}, integration with Facebook will be a major feature, as such this integration is subject to quality requirements in the form of:
		\begin{itemize}
			\item Reliability - The Facebook API server must be available at all times as this forms a major part of the system.
			\item Auditability - The use of the Facebook developer application ID allows for auditability as to the use of the Facebook API in our system. The developer application ID allows Facebook to obtain information on the current use of their API with regards to our system.
			\item Performance - Our system's performance in certain areas, is limited to the use of the Facebook API. The API system must respond before our system can analyse the data.
			\item Security - Facebook's API has security measures in place to ensure that the user's data isn't accessed without their consent. This consent is granted by means of a permission request to the user before we can obtain any information. The safeguard of this information is a major concern as mentioned in \ref{subsubsec:security}
			\item Maintainability - The system must be maintained on a bi-yearly basis as to ensure the latest version of the API is used and no deprecated functions are used. The use of deprecated functions can cause problems in future use of the system.
		\end{itemize}

	
	\subsection{Architecture Constraints}
	Our client limited our technologies to:
	\begin{itemize}
		\item An ASP.net web solution with a Microsoft SQL Server database system.
		\item A NodeJS web solution with a PostgreSQL database system.
		\item The use of a platform as a services (PaaS) hosting solution, Heroku.
		\item We have the freedom to use the Facebook SDK in any language except for PHP as mentioned below.
	\end{itemize}
	Our client specifically constrained the use of:
		\begin{itemize}
		\item PHP code in any way.
		\item MySQL, NoSQL database systems.
		\end{itemize}	
	Other constraints in terms of architecture include:
	\begin{itemize}
		\item Browser independence, to ensure any web-client can make use thereof.
		\item Operating system independence.
		\item The system should be as time efficient as possible to ensure a user gets a quote in less than 5 minutes.
		\end{itemize}
	
	
	\section{Initial Architecture Design}

	The below mentioned architecture specifies an initial architecture design from a high-level, and of the system as it is currently understood at this stage in development and design. Any and all architectural components, tactics and responsibilities listed are subject to change and are at a level which attempts to provide the best description of what is currently intended to be the insurance profiling system. 

	\subsection{Architectural tactics}

	Due to the fact that the underlying architecture has of yet not been clearly specified, the investigation into the best suited tactics for this architecture can not be concretely concluded. The following is by no means a representation of the implemented tactics but it is currently what best helps the system achieve its quality requirements. The below listed tactics are currently the main tactics we aim to implement or the responsibility of the dependencies we aim to utilise.
	
		\subsubsection{Modifiability and Flexibility}

		What is certain, is that the above mentioned quality requirements will need to be addressed in a manner which allows for a system that is both highly modular and configurable, in order to allow the analists of the data set to properly asses and tweak the tools that we aim to provide to them. 

		\begin{itemize}

			\item Flexibility support: automated builds, automated testing and automated environment configuration
			\item Service provider flexibility: dependency injection
			\item Process flexibility: pipes and filters, responsibility localisation

		\end{itemize}

		\subsubsection{Accessibility and Integratability}

		\begin{itemize}

			\item Providing access: providing proxies, supporting standard communication protocols (REST calls)

		\end{itemize}

		\subsubsection{Security}

		\begin{itemize}

			\item Resisting attacks: authentication, authorisation, minimizing access channels, minimize complexity and enforcing secure defaults
			\item Detecting attacks: auditing, logging and integrity checking
			\item Recovering: restore states, dropping

		\end{itemize}

		\subsubsection{Reliability}

		\begin{itemize}

			\item Preventing faults: resource locking, testing framework, contracts based
			\item Detect faults: logging, deadlock detection and error/exception communication
			\item Recovery: rollback, backups and passive redundancy

		\end{itemize}

		\subsection{Scalability}

		\begin{itemize}

			\item Resource demand: indexing, paging, query optimisation

		\end{itemize}

	\subsection{Architectural components addressing architectural responsibilities}

		\begin{itemize}

		%\item that of providing an environment for specifying and executing reports to a reporting frame-work,
		\item Providing human access channels and external systems web access to the system services which in turn grants access to a web services framework such as a custom REST api,
		\item Providing users access through a browser to a web application framework. Node.js has such a framework called express.js or .NETs ASP,
		\item In order to provide hosting for business logic processes, an application server will be hosted by Heroku,
		\item Persistence of objects will be stored in an object-relational database management system called PostgreSQL, or in the case of .NET, will be be managed by SQLSever,
		\item The responsibility of providing the persistence provider access to a persistence API such as node-postgres, or for .NET, an EntityFrameworkModel.


		\end{itemize}

	\subsection{Infrastructure}

	\subsubsection{Architectural (structural) patterns}

	In order to promote flexibility as well as testability, the pipes and filters pattern will be used. This allows for future development to change tactics (business logic tactics) and allows for the decoupling of different functions. This is crucial in order for the system to remain within its architectural concern boundaries, which will rely heavily on a modular configurable system. This, in conjunction with a layered architectural pattern (in order to add further access channels with regards to persistence other frameworks) will make for a understandable and testable system.

	\subsubsection{Concepts and constraints for application components}

	Application components will not be deployed at the first level of granularity. Any application components will be hosted within the architectural components.


	

\end{document}
