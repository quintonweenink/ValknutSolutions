\documentclass{article}

\usepackage[margin=2.5cm,left=2cm,includefoot]{geometry}
\usepackage{graphicx}
\usepackage{float}
\usepackage[space]{grffile}
\usepackage{hyperref}
\usepackage[export]{adjustbox}
\usepackage{multicol}
\usepackage{caption}
\usepackage{hyperref}
\usepackage{listings}

\usepackage{titlesec}

\setcounter{secnumdepth}{4}

\titleformat{\paragraph}
{\normalfont\normalsize\bfseries}{\theparagraph}{1em}{}
\titlespacing*{\paragraph}
{0pt}{3.25ex plus 1ex minus .2ex}{1.5ex plus .2ex}

% Header and footer
\usepackage{fancyhdr}
\pagestyle{fancy}

\rhead{COS301}
\lhead{Testing Document}
\fancyfoot[R]{Page \thepage}

\renewcommand{\headrulewidth}{2pt}
\renewcommand{\footrulewidth}{1pt}

\begin{document}

	\begin{titlepage}
		\begin{center}
			\includegraphics[width=10cm]{images/UP.jpg}  \\
			[0.5cm]
			\huge{
			Testing Document\\
			}
			
			\line(1,0){300}\\
			[0.2cm]
			\LARGE{Project: Insurance profiling from social media\\
			Client: RetroRabbit} \\
			\line(1,0){300}\\
			\LARGE{Team: Valknut Solutions}\\
			[1.0cm]
			\large
			{
			\begin{itemize}
				\item 13054903 - Charl Jansen van Vuuren 
				\item 10297902 - Bernhard Schuld      
				\item 13044924 - Kevin Heritage
				\item 13176545 - Quinton Weenink\\
			\end{itemize}
			}
			\textsc{\large}\\
		[3.0cm]
		\textsc{\large  Department of Computer Science}\\
		[0.5cm]
		\textsc{\large \today}\\
		\end{center}
			
	\end{titlepage}
	\cleardoublepage
	\tableofcontents
	\cleardoublepage
\section{Introduction}
This document contains information related to Testing of the Insurance Profiling project that is being developed for RetroRabbit as part of the COS301 module at the University Of Pretoria. \\
This document is based on the IEEE 829 Standard for Testing Documentation \href{http://www.fit.vutbr.cz/study/courses/ITS/public/ieee829.html}{IEEE 829}

\section{References to other documentation}
\begin{itemize}
	\item{Requirements specifications as per 29 July 2016}
	\item{Architecture Design as per 29 July 2016}
\end{itemize}

\section{Test items}
The items included in this test document includes:
\begin{itemize}
	\item Backend API calls
	\item Database management functions
	\item Environmental tests 
	\item Deployment and compatibility tests
\end{itemize} 

\section{Features to be Tested}
These include features to be tested are rated based on the level or risk associated namely, High (H), Medium (M) and Low (L).

\begin{itemize}
\item API calls to do:
	\begin{itemize}
	\item Database insertions (H)
	\item Database deletions (H)
	\item Database retrievals (L)
	\end{itemize}
\item Environmental test
	\begin{itemize}
	\item Integration test to determine the current environment (development or deployment) to react accordingly (M)
	\end{itemize}
\item Deployment/Build testing including Compatibility
		\begin{itemize}
	\item Automated testing by the Travis CI framework to ensure the current feature is compatible with different versions of the NodeJS framework. (H)
	\end{itemize}
\end{itemize}

\section{Features not to be Tested}
\begin{itemize}
	\item Email functionality
	\begin{itemize}
		\item This feature didn't form part of the previous sprint and is not yet in use.
		\item Testing is not yet needed or this feature
	\end{itemize}
	\item The ability to receive data from a Facebook lead ad cannot be tested but this data has been mocked to ensure proper integration testing.
	\item Front-end features including:
	\begin{itemize}
		\item Analyst login
		\item Authentication
	\end{itemize}
\end{itemize}

\section{Approach (Strategy)}
\begin{itemize}
	\item The testing framework Mocha will be used to run unit tests at compilation time or per request.
	\item Deployment and compatibility testing is done with the automated framework Travis CI. Travis ensures all changes made to a branch is compatible with the different versions of NodeJS and the current environment (development or deployment).
	\item A number of developed unit tests will be run with Mocha as can be seen in \ref{figurename}.
	\item The testing of Facebook lead data was mocked out for testing purposes.
	\item The use of Postman to test API routes manually, ie. to make POST requests. Postman is a browser extension developed for API 
\end{itemize}

\section{Item Pass/Fail Criteria for Tests}
If the specified Pre-condition is not met the test will Fail. \\
If the specified Post-condition is met the test will Pass
\begin{itemize}
\item Test creating row in Database - POST : /api/user/
	\begin{itemize}
	\item Pre-conditions:
		\begin{itemize}
		\item Database is connected and created
		\item The following values are specified:  
		\begin{itemize}
		\item firstName
  		\item lastName
  		\item mobileNumber 
  		\item maritalStatus 
  		\item dateOfBirth 
 		\item gender
  		\item location 
 		\item email
 		\end{itemize} 
		\end{itemize}
	\item Post-condition - The row gets persisted
	\end{itemize}
	
\item Test retrieving row in Database based on currentUser - GET : /api/user/:currentUser
	\begin{itemize}
	\item Pre-condition - The currentUser exists
	\item Post-condition - The currentUser is retrieved 
	\end{itemize}
	
\item Test removing row in Database based on currentUser - DELETE : /api/user/:currentUser
	\begin{itemize}
	\item Pre-condition - The currentUser exists
	\item Post-condition - The currentUser is removed 
	\end{itemize}	
\end{itemize}


\begin{figure}[H]
  \centering
      \includegraphics[width=\textwidth]{images/tests.png}
  \caption{Output of Mocha tests}
\end{figure}

\begin{figure}[H]
  \centering
      \includegraphics[width=\textwidth]{images/tests.png}
  \caption{Travis interface}
\end{figure}

\begin{figure}[H]
  \centering
      \includegraphics[width=\textwidth]{images/5_11.png}
  \caption{Output of Travis tests for Node version 5.11}
\end{figure}

\begin{figure}[H]
  \centering
      \includegraphics[width=\textwidth]{images/6_2.png}
  \caption{Output of Travis tests for Node version 6.2}
\end{figure}

\begin{figure}[H]
  \centering
      \includegraphics[width=\textwidth]{images/failed.png}
  \caption{Output of failed Travis test for illustration purposes}
\end{figure}



\end{document}
